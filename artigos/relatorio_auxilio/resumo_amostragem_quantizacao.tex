 % =======================================================================
% =                                                                     =
% =======================================================================
% -----------------------------------------------------------------------
% - Author:     Chaua Queirolo                                          -
% - Version:    001                                                     -
% -----------------------------------------------------------------------
\documentclass[a4paper,11pt]{article}    

% =======================================================================
% PACKAGES
% =======================================================================

% Language support
\usepackage[brazil]{babel}
\usepackage[utf8]{inputenc}
\usepackage[T1]{fontenc}
\usepackage{ae,aecompl}
\usepackage[section]{placeins}



% Configuration
\usepackage{url}
\usepackage{enumerate}
\usepackage{color}
\usepackage[svgnames,table]{xcolor}
\usepackage[margin=2cm,includefoot]{geometry}

% Tabular
\usepackage{multirow}
\usepackage{multicol}

% Images
\usepackage{graphicx}
\usepackage[scriptsize]{subfigure}
\usepackage{epstopdf}
\usepackage{float}% http://ctan.org/pkg/{multicol,lipsum,graphicx,float}


% Math
\usepackage{mdwtab}	% bug rowcolor
\usepackage{amssymb}
\usepackage{amsmath}
\usepackage{footnote}

% References
\usepackage[sort,nocompress]{cite}

% =======================================================================
% VARIABLES
% =======================================================================

% Space between the lines in a table
\renewcommand{\arraystretch}{1.3}

% Define a new column type
\newcolumntype{x}[1]{>{\raggedright\hspace{0pt}}p{#1}}%

% Controle das Margens
\sloppy
\tolerance=9999999

% Espaço entre colunas
\setlength{\columnsep}{.9cm}


% Configuration
\usepackage{lipsum}
\usepackage{blindtext}

% =======================================================================
% HEADER
% =======================================================================

\title{Resumo: Amostragem e Quantização de imagens}
\author{Alan Azevedo Bancks \\ Universidade Tuiuti do Parana \\E-mail: {\tt dsalan@hotmail.com}}
\date{03/05/2018}

\newenvironment{Figure}
  {\par\medskip\noindent\minipage{\linewidth}}
    {\endminipage\par\medskip}

% =======================================================================
% DOCUMENT
% =======================================================================
\begin{document}

\maketitle

%\begin{abstract}
%    \lipsum[1]
%\end{abstract}
%\hspace{.5cm}

\begin{multicols}{2}
\section{Resumo}
Quantização de uma imagem é um processamento digital para transformação de uma imagem continua para uma digital, seria algo como uma imagem real sendo processada digitalmente.Dois nomes são dados para operações que se seguem , amostragem para digitalização de valores na coordenada e quantização nos valores de amplitude.
O conceito de amostragem está diretamente relacionado ao zoom. Quanto mais amostras você recolher, mais pixels você tera.



\begin{figure}[H]
	\centering
	\includegraphics{imagem1}
	\caption{ (a)Uma imagem continua projeta em uma matriz, (b) Resultado de uma amostragem e quantização}
	\label{fig:nonfloat}
\end{figure}

A quantização é algo oposto à amostragem. Isso é feito no eixo y. Quando você está quantizando uma imagem, você está realmente dividindo um sinal em partes menores.
No eixo x do sinal, estão os valores de coordenadas e no eixo y, temos amplitudes. Portanto, digitalizar as amplitudes é conhecido como quantização.


Na figura abaixo (a) mostra uma imagem contínua f que queremos converter em formato digital, para isso temos que fazer a amostragem da função tanto nas coordenadas x e y quanto na amplitude. A digitalização dos valores de coordenadas é chamado de amostragem. A digitalização dos valores de amplitude é chamado de quantização.



Na figura (b) é apresentado um gráfico que representa os valores de amplitude da imagem contínua ao longo do segmento de reta AB da figura (a). Para realizar a amostragem foi colhido amostras igualmente espaçadas ao longo da linha AB, exibido na figura (c). A posição de cada amostra no espaço é indicada por uma pequena marca vertical na parte inferior da figura. O conjunto dessas localizações discretas nos dá a função de amostragem, porem os valores das amostras ainda cobrem uma faixa contínua de valores de intensidade.

Para formar uma função digital é necessário converter também os valores de intensidade (quantizados). O lado direito da figura (c) mostra uma escala de intensidade divido em oito intervalos discretos, variando do preto ao branco. As marcas verticais indicam o valor
específico atribuído a cada um dos oitos niveis de intensidade. Os niveis de intensidade
 continuos sao quantizados atribuindo um dos oito valores para cada amostra. Essa
 atribuicao e feita dependendo da proximidade vertical de uma amostra a uma marca
 indicadora. As amostras digitais resultantes da amostragem e da quantização pode ser
 vista na figura (d).

\begin{figure}[H]
	\centering 
	\includegraphics[width=\columnwidth]{imagem2}
	\caption{(a) imagem continua, (b) linha de varredura A-B, (c) Amostragem e quantização (d) linha de varredura digital}
	\label{fig:nonfloat}
\end{figure}


A figura quantizada mostrada abaixo tem 3 tons de cinza. Isso significa que a imagem formada a partir desse sinal teria apenas 5 cores diferentes. Seria uma imagem em preto e branco com algumas vaziações de cinza. Agora, se você fosse para melhorar a qualidade da imagem, há uma coisa que pode ser feita que para aumentar os níveis, ou aumentar o tom de cinza. Se você aumentar este nível para 256, significa que você tem uma imagem em escala de cinza. Que é muito melhor do que simples imagem em preto e branco.


\begin{figure}[H]
	\centering
	\includegraphics{nivelcinza1}
	\caption{imagem de 8bpp, que tem 256 níveis diferentes onde em tons de cinza e a imagem se parece com algo assim.}
	\label{fig:nonfloat}
\end{figure}

\begin{figure}[H]
	\centering
	\includegraphics{nivelcinza2}
	\caption{reduzindo os níveis de cinza de 256 para 128.}
	\label{fig:nonfloat}
\end{figure}

\begin{figure}[H]
	\centering
	\includegraphics{nivelcinza3}
	\caption{
		Agora de 128 para 64}
	\label{fig:nonfloat}
\end{figure}




%Exemplo de citação~\cite{ref:gonzales2011}.


\bibliographystyle{plain}
\bibliography{referencias}

\end{multicols}
\end{document}

